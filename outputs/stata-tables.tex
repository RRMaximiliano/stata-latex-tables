\documentclass{article}
%-----------------------------------------------------------------
% PREAMBLE -------------------------------------------------------
%-----------------------------------------------------------------
\usepackage[flushleft]{threeparttable}						
%\usepackage[nolists,heads]{endfloat}
\usepackage{pdflscape}
\usepackage{amssymb}
\usepackage{dcolumn}
\usepackage{multirow}
\usepackage{longtable}
\usepackage{booktabs}
\usepackage{setspace}
\usepackage[skip = 0pt]{caption}
\captionsetup{justification=centering}
\usepackage{subcaption}
\usepackage{footnote}
\usepackage{fullpage}
\usepackage{mathrsfs,amsfonts}
\usepackage{amsmath}
\usepackage{graphicx}
\usepackage{float}
\usepackage{changepage}
\usepackage{tabularx}
\usepackage{siunitx}
\usepackage[table]{xcolor}
\definecolor{airforceblue}{rgb}{0.36, 0.54, 0.66}
\definecolor{olivine}{rgb}{0.6, 0.73, 0.45}
\usepackage{array}
\usepackage[plainpages=false,pdfpagelabels]{hyperref}
\usepackage[english]{babel}
\usepackage[utf8]{inputenc}
\usepackage{nameref}
\usepackage [autostyle, english = american]{csquotes}
\usepackage{apacite}
\usepackage{rotating}
\usepackage[T1]{fontenc}
\usepackage{lscape}
\usepackage{adjustbox}
\usepackage{afterpage}
\usepackage{booktabs}
\usepackage{lipsum}
%-----------------------------------------------------------------
%font
%-----------------------------------------------------------------
\usepackage[sc]{mathpazo}
%-----------------------------------------------------------------
%DAG
%-----------------------------------------------------------------
\usepackage{tikz}
\usetikzlibrary{positioning}
%\tikzset{mynode/.style={draw,text width=1in,align=center}}
\tikzset{mynode/.style={draw,align=center}}
%-----------------------------------------------------------------
\makeatletter\let\expandableinput\@@input\makeatother
\MakeOuterQuote{"}
\setcounter{MaxMatrixCols}{10}
\newtheorem{acknowledgement}{Acknowledgement}
\newtheorem{algorithm}{Algorithm}
\newtheorem{axiom}{Axiom}
\newtheorem{case}{Case}
\newtheorem{claim}{Claim}
\newtheorem{conclusion}{Conclusion}
\newtheorem{condition}{Condition}
\newtheorem{conjecture}{Conjecture}
\newtheorem{corollary}{Corollary}
\newtheorem{criterion}{Criterion}
\newtheorem{definition}{Definition}
\newtheorem{example}{Example}
\newtheorem{exercise}{Exercise}
\newtheorem{lemma}{Lemma}
\newtheorem{notation}{Notation}
\newtheorem{problem}{Problem}
\newtheorem{proposition}{Proposition}
\newtheorem{remark}{Remark}
\newtheorem{solution}{Solution}
\newtheorem{assumption}{Assumption}
%-----------------------------------------------------------------
% new environment for landscape tables    
\newenvironment{ltable}{\begin{landscape}\begin{table}}{\end{table}\end{landscape}}
\newenvironment{ltablelong}{\begin{landscape}\begin{longtable}}{\end{longtable}\end{landscape}}
%-----------------------------------------------------------------
\newcolumntype{H}{>{\setbox0=\hbox\bgroup}c<{\egroup}@{}}
\newcolumntype{P}[1]{>{\centering\arraybackslash}p{#1}}
%-----------------------------------------------------------------
\newcommand\independent{\protect\mathpalette{\protect\independenT}{\perp}}
\def\independenT#1#2{\mathrel{\rlap{$#1#2$}\mkern2mu{#1#2}}}

%-----------------------------------------------------------------
\makeatletter
\newcommand\primitiveinput[1]
{\@@input #1 }
\makeatother
%-----------------------------------------------------------------
\def\sym#1{\ifmmode^{#1}\else\(^{#1}\)\fi}
%-----------------------------------------------------------------
% tables numbers setup
%\numberwithin{table}{section}
%-----------------------------------------------------------------
% colors
%-----------------------------------------------------------------
\usepackage{colortbl}
\usepackage{url}
\urlstyle{rm}
\definecolor{darkblue}{rgb}{0,0,.4}
\hypersetup{colorlinks=true, 
			breaklinks=true, 
			citecolor=darkblue, 
			linkcolor=darkblue, 
			menucolor=darkblue, 
			urlcolor=darkblue}
%-----------------------------------------------------------------
\begin{document}
%-----------------------------------------------------------------
\title{Stata Tables}
\author{Rony Rodriguez-Ramirez\\{\normalsize The World Bank, Development Impact Evaluation Unit}}
\maketitle
\listoftables
%-----------------------------------------------------------------
\newpage
%-----------------------------------------------------------------
% FIRST TABLE
%-----------------------------------------------------------------
\begin{table}[H]
	\centering
	\label{tab:Table}
	\begin{adjustbox}{max width=\linewidth}
		\begin{threeparttable}
			\caption{Basic exported table}
			\begin{tabular}{@{}l*{4}{c}@{}}
                \toprule
                \toprule 
				& (1) & (2) & (3) & (4) \\
				\primitiveinput{tables/t1_basic.tex}
				\bottomrule
			\end{tabular}
			\begin{tablenotes}
				\setlength\labelsep{0pt}
				\footnotesize
				\item \textit{Notes}: \lipsum[1].
			\end{tablenotes}
		\end{threeparttable}
	\end{adjustbox}
\end{table}
%-----------------------------------------------------------------
% SECOND TABLE
%-----------------------------------------------------------------
\begin{table}[H]
	\centering
	\label{tab:Table}
	\begin{adjustbox}{max width=\linewidth}
		\begin{threeparttable}
			\caption{Adding variable labels}
			\begin{tabular}{@{}l*{4}{c}@{}}
                \toprule
                \toprule 
				& (1) & (2) & (3) & (4) \\
				\primitiveinput{tables/t2_labels.tex}
				\bottomrule
			\end{tabular}
			\begin{tablenotes}
				\setlength\labelsep{0pt}
				\footnotesize
				\item \textit{Notes}: \lipsum[1].
			\end{tablenotes}
		\end{threeparttable}
	\end{adjustbox}
\end{table}
%-----------------------------------------------------------------
% THIRD TABLE
%-----------------------------------------------------------------
\begin{table}[H]
	\centering
	\label{tab:Table}
	\begin{adjustbox}{max width=\linewidth}
		\begin{threeparttable}
			\caption{Removing base category}
			\begin{tabular}{@{}l*{4}{c}@{}}
                \toprule
                \toprule 
				& (1) & (2) & (3) & (4) \\
				\primitiveinput{tables/t3_labels_omitted.tex}
				\bottomrule
			\end{tabular}
			\begin{tablenotes}
				\setlength\labelsep{0pt}
				\footnotesize
				\item \textit{Notes}: \lipsum[1]. 
			\end{tablenotes}
		\end{threeparttable}
	\end{adjustbox}
\end{table}
%-----------------------------------------------------------------
% Fourth Table
%-----------------------------------------------------------------
\begin{table}[H]
	\centering
	\label{tab:Table}
	\begin{adjustbox}{max width=\linewidth}
		\begin{threeparttable}
			\caption{Adding fixed effects}
			\begin{tabular}{@{}l*{4}{c}@{}}
                \toprule
                \toprule 
				& (1) & (2) & (3) & (4) \\
				\primitiveinput{tables/t4_scalars.tex}
				\bottomrule
			\end{tabular}
			\begin{tablenotes}
				\setlength\labelsep{0pt}
				\footnotesize
				\item \textit{Notes}: \lipsum[1].
			\end{tablenotes}
		\end{threeparttable}
	\end{adjustbox}
\end{table}
%-----------------------------------------------------------------
% FIFTH TABLE
%-----------------------------------------------------------------
\begin{table}[H]
	\centering
	\label{tab:Table}
	\begin{adjustbox}{max width=\linewidth}
		\begin{threeparttable}
			\caption{Complete table with model titles and notes}
			\begin{tabular}{@{}l*{4}{c}@{}}
                \toprule
                \toprule 
                &\multicolumn{4}{c}{Dependent variable:} \\
                \cmidrule(lr){2-5} 
                &\multicolumn{2}{c}{Number of Deaths} & \multicolumn{2}{c}{Number of Divorces}  \\ 
                \cmidrule(r){2-3}
                \cmidrule(l){4-5}  
                & (1) & (2) & (3) & (4) \\
                \primitiveinput{tables/t4_scalars.tex}
                \bottomrule
			\end{tabular}
			\begin{tablenotes}
				\setlength\labelsep{0pt}
				\footnotesize
				\item \textit{Notes}: \lipsum[1].
			\end{tablenotes}
		\end{threeparttable}
	\end{adjustbox}
\end{table}
%-----------------------------------------------------------------

\end{document}
%-----------------------------------------------------------------